\chapter{Título del Apéndice 1}
\label{apx:1}

\section{Algoritmo WWW}
\label{sec:WWW}

\noindent Ejemplo de código con coloreado de sintaxis.

\begin{lstlisting}[language=C++]
#include <iostream>

int main()
{
  // Imprime "Hello, world!" en la consola
  std::cout << "Hello, world!\n";
  return 0;
}
\end{lstlisting}

\section{Archivo XXX}
\label{sec:XXX}

\noindent Ejemplo de JSON usando el mismo entorno de coloreado de sintaxis.

\begin{lstlisting}
{
    "nombre": "John Doe",
    "edad": 30,
    "ciudad": "Nueva York",
    "hobbies": [
        "lectura",
        "jardinería",
        "ciclismo"
    ],
    "empleo": {
        "título": "Ingeniero de Software",
        "empresa": "TechCorp"
    }
}
\end{lstlisting}
 
\section{Algoritmo YYY}
\label{Apendice1:YYY}

\noindent El clásico entorno \texttt{verbatim}, sin coloreado pero con fuente monoespaciada.

\begin{center}
\begin{footnotesize}
\begin{verbatim}

/***********************************************************************************
 *
 * Fichero .h
 *
 ***********************************************************************************
 *
 * AUTORES
 *
 * FECHA
 *
 * DESCRIPCION
 *
 *
 ************************************************************************************/
 
\end{verbatim}
\end{footnotesize}
\end{center}

\section{Algoritmo ZZZ}
\label{Apendice1:ZZZ}

\noindent Ejemplo de entorno para describir algoritmos en pseudocódigo.

\begin{algorithm}[htpb]
    \caption{Cálculo del factorial de un número}\label{alg:factorial}
    \begin{algorithmic}[1]
        \State \textbf{Entrada:} Un entero $n$
        \State \textbf{Salida:} El factorial de $n$
        \Function{Factorial}{$n$}\Comment{El factorial de n}
            \If{$n \leq 1$}
                \State \Return $1$\Comment{El factorial de 0 o 1 es 1}
            \Else
                \State \Return $n \times$ \Call{Factorial}{n-1}
            \EndIf
        \EndFunction
    \end{algorithmic}
\end{algorithm}

Otra forma de describir algoritmos es utilizar entornos \texttt{lstlisting} y emplear una sintaxis de pseudocódigo similar a alguno de los lenguajes soportados por este paquete, como Python o Pascal.